% Fazit und Ausblick
\chapter{Fazit und Ausblick}
In dieser Studienarbeit wurden elliptische Kurven in der Charakteristik $p > 3$ untersucht. Die erläuterten Grundlagen in Kapitel \ref{sec:grundlagen} halfen dabei, ein Verständnis der mathematischen Aufzubauen. Die Definition und Arithmetik wurden anschaulich in Kapitel \ref{sec:elliptische_kurve} erläutert. Neben drei Algorithmen zur Punktbestimmung wurde ebenfalls das diskrete Logarithmusproblem auf elliptischen Kurven sowie der Diffie-Hellmann-Schlüsselaustausch auf elliptischen Kurven je mittels eines Beispiels erklärt. Im letzten Kapitel \ref{sec:python} wurde die Implementierung der Arithmetik und der Rechengesetze in Python implementiert. Weiterhin wurden Hilfsalgorithmen sowie der DHKE über elliptischen Kurven programmiert.\\

Bei den gewählten Beispielen wurden verschiedenste elliptische Kurven und Primzahlen aus unterschiedenen Quellen verwendet. Innerhalb dieser Studienarbeit wurde kein Algorithmus oder eine andere Möglichkeit gezeigt, wie elliptische Kurven gefunden oder generiert werden können. Dies könnte in einer künftigen Arbeit noch ergänzt werden. Es könnte zum Beispiel ein oder mehrere Verfahren vorgestellt werden, welche helfen die Parameter $a$, $b$ und $p$ geeigneter elliptischer Kurve zu berechnen. Im Teil der Grundlagen könnte man das Unterkapitel um die Bestimmung von Primzahlen ergänzen, indem man die Riemannsche Zeta-Funktion $\zeta (x)$ hinzufügt und kurz die Riemannsche Vermutung thematisiert. Man könnte in einer fortsetzenden Studienarbeit die Umsetzung der Arithmetik und der Rechengesetze von elliptischen Kurven in einer anderen Programmiersprache umzusetzen. Beispielsweise könnte man eine maschinennahe Sprache wie C wählen welche kompiliert wird, damit man die Rechnungen effizienter gestalten kann als in einer Programmiersprache wie Python, welche interpretiert werden muss. Weiterhin könnte man auf die Geschichte der elliptischen Kurven und mehr auf deren Nutzen in der Kryptographie eingehen. Eine weitere mögliche Ergänzung, wäre es genauer auf kryptografische Protokolle unter der Verwendung elliptischer Kurven einzugehen und diese aufbauend auf dem schon bestehenden Programm zu implementieren.