%Elliptische Kurven


\chapter{Elliptische Kurven}
Als Basis für Asymentrische Kryptosysteme können elliptische Kurven dazu genutzt werden, die verschlüsselungstechnische Effektivität mathematischer Probleme, wie das des diskreten Logarithmus, zu erhöhen. Bei der Kryptographie unter Verwendung elliptsicher Kurven bei deutlich kürzerer Schlüssellänge ein gleichwertigers Ergebnis erzielt werden. Dieser Effekt wird durch die spezielle Arithmetik auf elliptischen Kurven erzielt, deren mathematische Grundlage, konkrete Eingenschaften und Funktionsweise im folgenden Kapitel erörtert werden soll.

Elliptische Kurven können über beliebigen Körpern definiert werden. Für die Kryptographie interessant sind elliptische Kurven über Primkörpern.

%\section{Definition: Elliptische Kurven}
% Sei \textit{p} eine \textit{Primzahl} $p>3$. Seien $a,b \in GF(p)$. Betrachte die Gleichung
%\begin{equation}
%    y^2 z = x^3 + axz^2 + bz^3.\label{eq:g1}
%\end{equation} Die \textit{Diskriminante}  dieser Gleichung ist
%\begin{equation}
%    \delta = -16(4a^3 + 27b^2).\label{eq:g2}
%\end{equation}
%Wir nehmen an, dass die Diskriminante $\delta$ nicht Null ist. Ist $(x,y,z) \in GF(p)^3$ eine Lösung dieser
%Gleichung, so ist für alle $c \in GF(p)$ auch $c(x,y,z)$ eine solche Lösung. Zwei Lösungen $(x,y,z)$ und $(x',y',z')$
%heißen \textit{äquivalent}, wenn es ein von Null verchiedenes $c \in GF(p)$ gibt mit $(x,y,z) = c(x',y',z')$. Dies
%definiert eine Äquivalenzrelation auf der Menge aller Lösungen der~\eqref{eq:g1}.




Um das weitere Verständnis zu verbessern, wollen wir uns erst eine uns schon bekannte Kurve ansehen. In Abbildung XY ist das Polynom $x^2 + y^2 = r^2$ über $\mathbb{R}$ dargestellt. Wie zu sehen ist, handelt es sich hierbei um die Kreisgleichung. Der zu sehende Kreis ist nichts anderes als die Menge aller Punkte, welche die Kreisgleichung erfüllen.
Ein Beispiel für eine solchen ist der Punkt $(r,0)$. Wenn $x$ den Wert $r$ hat, muss $y$ folglich den Wert $0$ haben. Ein Gegenbeispiel ist der Punkt $(r,r/2)$. Dieser erfüllt die Kreisgleichung nicht.
\begin{figure}[!h]
    \centering
    \includegraphics[width=0.3\textwidth]{grafiken/Kreis.PNG}
    \caption[r]{r}
    \label{fig:aufgaben_redesign}
\end{figure}
Die Kreisgleichung kann verallgemeinert werden, indem den Termen $x^2$ und $y^2$ Koeffizienten voran gesetzt werden. Eine solche Gleichung, $ax^2 + by^2 = c$ erzeugt über $\mathbb{R}$ eine Ellipse, wie in Abbildung XY zu sehen.
\begin{figure}[H]
    \centering
    \includegraphics[width=0.3\textwidth]{grafiken/Ellipse.PNG}
    \caption[r]{r}
    \label{fig:aufgaben_redesign}
\end{figure}
Eine elliptische Kurve ist nun eine spezielle Polynomgleichung, der Form $y^2 = x^3 + ax + b$, unter der Bedingung $4a^3 + 27b^3 \neq 0$. Eine solche Gleichung über $\mathbb{R}$ ist in Abbildung XY dargestellt.
\begin{figure}[!h]
    \centering
    \includegraphics[width=0.3\textwidth]{grafiken/Elliptische_Kurve.PNG}
    \caption[r]{r}
    \label{fig:XXXX}
\end{figure}
Damit elliptische Kurven sinnvoll in der Kryptologie eingesetzt werden können, muss die Polynomgleichung über einem Primkörper betrachtet werden. Das heißt einfach gesprochen, alle Berechnungen werden modulo $p$ durchgeführt.
\paragraph{Definition: Elliptische Kurven über Primkörpern}
Die \textit{elliptische Kurve} über $\mathbb{F_p}$, ist die Menge aller Punkte $(x,y)$ mit $x,y \in \mathbb{F_p}$, welche die folgende Gleichung erfüllen: 
\begin{center}
$y^2 \equiv x^3 + ax + b$ mod $p$, wobei $a,b \in \mathbb{F_p}$
\end{center} 
und die Bedingung  $$4a^3 + 27b^3 \neq 0$$ gelten müssen. Zu der elliptischen Kurve gehört des Weiteren auch der imaginäre \textit{Punkt im Unendlichen} $\mathcal{O}$.

Durch die Bedingung XY werden sog. Singularitäten ausgeschlossen. Andernfalls gäbe es Punkte, deren Tangente nicht wohldefiniert ist, was für das Rechnen auf elliptischen Kurven jedoch erforderlich ist.

Nachdem elliptische Kurven nun definiert wurden, stellt sich die Frage, wie diese nun in der Kryptographie eingesetzt werden können. Wenn wir uns an das in Kapitel XY zurückerinnern, wird für die Konstruktion eines \textbf{DLP}s eine zyklische Gruppe benötigt. Eine eben solche findet sich in der Punktmenge der elliptischen Kurve wieder. Offen bleibt wie die Gruppenoperation definiert ist. Diese muss die in Kapitel XY geforderten Gruppengesetze erfüllen.

Als Symbol für die Gruppenoperation wird das Additionszeichen $+$ verwendet. Durch die Gruppenoperation muss aus zwei Punkten $P = (x_1, y_2)$ und $Q= (x_2, y_2)$ der Kurve ein dritter Punkt $R$ auf der Kurve berechnet werden. 
$$P + Q = R$$ $$(x_1, y_1) +  (x_2, y_2) = (x_3, y_3)$$
Am verständlichsten lässt sich diese Operation grafisch zeigen.

Elliptische Kurven über endlichen Körpern können grafisch nicht sinnvoll dargestellt werden. Ihre Form und Arithmetik lassen sich jedoch gut veranschaulichen wenn man sie auf $\mathbb{R}$ abbildet. Im Folgenden betrachten wir eine Elliptische Kurve, dargestellt in einem kartesischen Koordinatensystem, um die Gruppeneigenschaften bezüglich der Punktaddition zu zeigen. Hierbei sind nun zwei Fälle zu unterscheiden. 

\textbf{Punktaddition $P + Q$:}
Falls $P \neq Q$ erfolgt die geometrische Konstruktion, indem zunächst eine Grade durch die beiden Punkte gelegt wird. Aufgrund der Kurveneigenschaften hat diese immer einen dritten Schnittpunkt mit der Kurve. Dieser wird an der $x$-Achse gespiegelt um den gesuchten Punkt $R$ zu erhalten. Abbildung XY zeigt die beschriebene Konstruktion.

\begin{figure}[H]
    \centering
    \includegraphics[width=0.3\textwidth]{grafiken/Punktaddition.PNG}
    \caption[Punktaddition]{r}
    \label{fig:Punktaddition}
\end{figure}

\textbf{Punktverdopplung $P + P$:}
Falls P und Q identisch sind erfolgt die geometrische Konstruktion, indem eine Tangente an den Punkt $P$ angelegt wird. Diese liefert wieder einen weiteren Schnittpunkt mit der Kurve, welcher an der $x$-Achse gespiegelt wird um den Punkt $R$ zu erhalten. Anstatt $R = P + Q$ schreibt man in diesem Fall $R = P + P = 2P$   Abbildung XY zeigt die beschriebene Konstruktion. 

\begin{figure}[H]
    \centering
    \includegraphics[width=0.3\textwidth]{grafiken/Punktverdopplung.PNG}
    \caption[Punktverdopplung]{r}
    \label{fig:Punktverdopplung}
\end{figure}

Nach dieser grafischen Veranschaulichung sollte es leichter fallen die folgenden Formeln für die Punktaddition bzw. Punktverdopplung nachvollziehen zu können. Die Gruppenoperation existiert in jedem Körper, weshalb die Berechnung von $R$, wie grade gezeigt über den reellen Zahlen $\mathbb{R}$, als auch über einem Primkörper $\mathbb{F_p}$ durchgeführt werden kann.

\textbf{Formel: Punktaddition und -verdopplung auf elliptischen Kurven:}
$$x_3 = s^2 - x_1 - x_2$$
$$y_3 = s(x_1 - x_2) - y_1$$,
wobei

$$s = \begin{cases}
	\frac{y_2 - y_1}{x_2 -x_1} & \text{, falls } P \neq Q \text{ (Punktaddition)}\\
	\frac{3x_1^2 + a}{2y_1} & \text{, falls } P = Q \text{ (Punktverdopplung)}
	\end{cases}
$$
 



























