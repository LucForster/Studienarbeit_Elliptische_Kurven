\chapter{Kryptosysteme unter Verwendung elliptischer Kurven}
\section{Elliptic-Curve-Diffie-Hellman Key Exchange (ECDHKE)}
Der ECDHKE ist eine Anpassung des klassischen DHKE, wobei hier als zyklische Gruppe nicht $\mathbb{Z}_p^*$ sondern die Punktemenge einer elliptischen Kurve verknüpft durch die in Kapitel \ref{sec:Arithmetik} definierte Gruppenoperation. Wie in Kaptiel \ref{sec:DLPüberEC} gezeigt, kann auch in einer solchen zyklischen Gruppe bzw. Untergruppe ein DLP definiert werden, welches sogar deutlich schwieriger zu lösen ist, als das DLP über $\mathbb{Z}_p^*$.

Die Konstruktion des ECDHKE funktioniert analog zu jener des DHKE, welche in Kapitel \ref{sec:DHKE} erläutert wurde. zunächst müssen wieder die Domain-Parameter festgelegt werden:

\paragraph{ECDHKE-Domain-Parameter}
\begin{enumerate}
\item Wähle eine Primzahl $p$ und eine elliptische Kurve $$E: y^2 \equiv x^3 + a \cdot x +b \text{ mod } p$$
\item Wähle ein primitives Element $P = (x_P, y_P)$
\end{enumerate}
Die Primzahl $p$, die Kurvenkoeffizienten $a$, $b$ und der Punkt $P$ bilden die sog. Domain-Parameter.

Die Wahl dieser Parameter hat großen Einfluss auf die Anzahl der Gruppenelemente und damit auf die Sicherheit des Protokolls. Die Bestimmung geeigneter Parameter ist relativ rechenaufwändig, weshalb dies in der Praxis nicht jedes mal neu gemacht wird, sondern z.B. von der NIST vorgegebene Parameter verwendet werden. Der Schlüsselaustausch erfolgt nun analog zum DHKE über endlichen Körpern:

\paragraph{Diffie-Hellman-Schlüsselaustausch mit elliptischen Kurven (ECDH)}
\begin{tabbing}
\qquad \qquad \qquad \qquad \qquad \qquad \qquad \qquad \= \qquad \qquad \qquad \= \qquad \qquad \qquad \qquad \qquad \qquad \qquad \qquad \kill
\textbf{Alice} \> \> \textbf{Bob}\\
Wähle $a=k_{pr,A} \in \{2,..., \#E-1\}$ \> \> Wähle $b=k_{pr,B} \in \{2,..., \#E-1\}$\\
Berechne $k_{pub,A} = aP = A = (x_A,y_A)$ \> \> Berechne $k_{pub,B} = bP = B = (x_B, y_B)$\\
\> $\xleftarrow{k_{pub,A} = A}$ \> \\
\> $\xrightarrow{k_{pub,B} = B}$ \> \\
Berechne $aB = T_{AB}$ \> \> Berechne $bA = T_{AB}$\\
Gemeinsames Geheimnis zwischen Alice und Bob: $T_{AB} = (x_{AB}, y_{AB})$\\
\end{tabbing}
Das Alice und Bob den gleichen Punkt berechnet haben lässt sich einfach zeigen.\\

Alice berechnet: $$aB = a(bP)$$
Bob berechnet: $$bA = b(aP)$$

Da die Punktaddition assoziativ ist, berechnen Alice und Bob den gleichen Punkt $T_{AB} = abP$.\\

Alice und Bob wählen zunächst jeweils eine große zufällige Primzahl $a$ und $b$, welche als private Schlüssel dienen. Beide multiplizieren ihren privaten Schlüssel mit dem durch die Domain-Parameter gegebenen Generatorpunkt $P$ um die öffentlichen Schlüssel $A$ und $B$ zu erhalten. Diese werden gegenseitig ausgetauscht. Im folgenden multiplizieren beide ihren privaten Schlüssel ($a$ bzw. $b$) mit dem empfangenen öffentlichen Schlüssel ($B$ bzw. $A$) des anderen um $T_{AB}$ zu erhalten. Aus diesem gemeinsamen Geheimnis kann nun ein Schlüssel für eine symmetrische Verschlüsselung abgeleitet werden. In der Regel wird dazu der Wert der $x$-Koordinate verwendet. Folgend wird der ECDHKE beispielhaft mit kleinen Zahlen durchgeführt.\\

\begin{tabbing}
\qquad \qquad \qquad \qquad \qquad \qquad \qquad \qquad \= \qquad \qquad \qquad \= \qquad \qquad \qquad \qquad \qquad \qquad \qquad \qquad \kill
\textbf{Alice} \> \> \textbf{Bob}\\
Wähle $a=k_{pr,A} = 3$ \> \> Wähle $b=k_{pr,B} = 10$\\
$k_{pub,A} = A = 3P = (10,6)$ \> \> $k_{pub,B} = B = 10P = (7,11)$\\
\> $\xleftarrow{k_{pub,A} = A}$ \> \\
\> $\xrightarrow{k_{pub,B} = B}$ \> \\
$T_{AB} = aB = 3(7,11) = (13,10)$ \> \> $T_{AB} = bA = 10(10,6) = (13,10)$\\
Gemeinsames Geheimnis zwischen Alice und Bob: $T_{AB} = (13,10)$\\
\end{tabbing}
Für die Punktmultiplikationen wird der in Kapitel \ref{sec:DLPüberEC} erläuterte Double-and-Add-Algorithmus verwendet.



\section{El-Gamal Verschlüsselung}