% Abstract
%\begin{footnotesize}
\selectlanguage{ngerman}
\begin{abstract}
\thispagestyle{plain}
In dieser Studienarbeit wurden elliptisch Kurven in der Charakteristik mit $p > 3$ untersucht. Dafür wurde durch ein Kapitel mit Grundlagen wichtige Vorkenntnisse erläutert die zum Verständnis der nachgehenden Kapitel wichtig sind. Dazu gehört Grundlagenwissen über Primzahlen und deren Bestimmung sowie über algebraische Stukturen wie beispielsweise Gruppen und Körper. In diesem Zuge wurde auch der erweiterte euklidische Algorithmus erklärt, der für das Berechnen des Inversen wichtig ist. Im Anschluss wurde das diskrete Logarithmusproblem und dazugehörig der Diffie-Hellman Key Exchange erklärt. Es wurde die Arithmetik sowie die Gruppeneigenschaften der elliptischen Kurven erklärt. Dazu zählen auch die Rechengesetze auf elliptischen Kurven. Im Anschluss wurden drei Möglichkeiten zur Bestimmung der Punkte auf elliptischen Kurven mit ausführlichen Beispielen erläutert. Danach wurde das diskrete Logarithmusproblem über elliptischen Kurven und der Elliptic-Curve-Diffie-Hellman Key Exchange über Beispiele erläutert. In nächsten Kapitel wurden einige in der Studienarbeit vorkommenden Algorithmen sowie die Arithmetik der elliptischen Kurven und der Diffie-Hellman Key Exchange in Python programmiert. Die Funktionsweise der Codeabschnitte wurde jeweils erklärt. Am Ende der Arbeit wurden mögliche Anknüpfungspunkte an zukünftige Studienarbeiten erwähnt. Künftige Arbeiten können sich beispielsweise darauf fokussieren Algorithmen oder andere Möglichkeiten zu finden, um elliptische Kurven zu generieren oder näher auf die Bestimmung von Primzahlen eingehen. Weiterhin wurde thematisiert, dass eine zukünftige Studienarbeit die elliptischen Kurven mit einer anderen Programmiersprache umsetzen könnte wie die maschinennahe Programmiersprache C. Es wurde auch vorgeschlagen, dass man in einer Anschlussarbeit näher auf die Geschichte der elliptischen Kurven und deren Nutzen in der Kryptographie eingehen kann. Ein konkretes Eingehen auf kryptographische Protokolle wurde ebenfalls thematisiert.
\end{abstract}

\selectlanguage{english}
\begin{abstract}
\thispagestyle{plain}
In this thesis elliptic curves with $p > 3$ were investigated. For this purpose, a chapter with basics explained important preliminary knowledge which is important for the understanding of the following chapters. This includes basic knowledge about prime numbers and their determination as well as about algebraic structures like groups and solids. In this course the extended Euclidean algorithm was explained, which is important for the calculation of the inverse. Subsequently, the discrete logarithm problem and the associated Diffie-Hellman key exchange were explained. The arithmetic as well as the group properties of the elliptic curves were explained. This also includes the calculation laws on elliptic curves. Subsequently, three ways of determining points on elliptic curves were explained with detailed examples. After that, the discrete logarithm problem over elliptic curves and the Elliptic-Curve-Diffie-Hellman Key Exchange were explained via examples. In next chapter, some algorithms appearing in the student research paper as well as the arithmetic of elliptic curves and Diffie-Hellman Key Exchange were programmed in Python. The functionality of the code sections was explained in each case. At the end of the paper, possible links to future student research were mentioned. Future work could, for example, focus on finding algorithms or other ways to generate elliptic curves or go into more detail on the determination of prime numbers. Furthermore, it was discussed that a future student research project could implement the elliptic curves with another programming language like the machine-oriented programming language C. It was also suggested that a follow-up paper could go into more detail about the history of elliptic curves and their usefulness in cryptography. A concrete discussion of cryptographic protocols was also suggested.
\end{abstract}
\selectlanguage{ngerman}
%\end{footnotesize}