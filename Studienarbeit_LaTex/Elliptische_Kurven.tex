\documentclass[12pt]{article}							%Schriftgröße und Art des Dokuments
\usepackage[ngerman]{babel}								%Deutsche Rechtschreibung
\usepackage[utf8]{inputenc}								%Codierung (wichtig für Umlaute)
\usepackage[dvipsnames]{xcolor}							%Für farbigen Text
\usepackage[a4paper,lmargin={3cm},rmargin={3cm},		%Seitenränder und Format
tmargin={2.5cm},bmargin = {2cm}]{geometry}		
\usepackage{amssymb}									%für mathematische Symbole
\usepackage{amsthm}										%für mathematische Umgebungen
\usepackage{graphicx} 									%für Grafiken
\usepackage{booktabs} 									%für Tabellen
\usepackage{blindtext} 									%für Bildtext
\usepackage{textcomp} 									%für besondere Symbole wie z.B €
\usepackage{subfig}										%für Platzierung von Grafiken etc.
\usepackage{float}										%für Platzierung von Grafiken etc.
\usepackage{setspace}
\usepackage{listings}									%für Programmcode 
\usepackage{acronym} 									%für Abkürzungsverezichnis
\usepackage{relsize}
\usepackage{enumitem}


\lstset{ 												% General setup for the listings package
    language=C++,
    basicstyle=\small \ttfamily,%\sffamily,
    breaklines=true,
    breakautoindent=true,
    numbers=left,
    %numberstyle=\tiny,
    frame=tb,
    tabsize=4,
    columns=fixed,
    showstringspaces=false,
    showtabs=false,
    keepspaces,
    commentstyle=\color{PineGreen},
    keywordstyle=\color{Blue}
}

\linespread{1.25}										%Zeilenabstand
\pagestyle{headings}

\title{T3\_2000 Praxisbericht\\Umstellung der Schlüsselverwaltung für die ECU Provisionierung von einer lokalen Schlüssel-Datenbank zu einer den Schlüssel extern einbringenden Methode} 	%Titel
\author{Luc Forster}									%Autor
\date{\copyright\today}									%Datum und Copyright

%\hyphenation{Schlüs-sel}

\bibliographystyle{unsrt}								%Zitationsstil

%%% Für Quellcodeverzeichnis %%%
\makeatletter
 % Festlegen des Kapitelnamen (nicht unbedingt notwendig):
 % aus listing.sty

 
 
%%%%%%%%%%%%%%%%%%%%%%%%%%%%%%%%%%%%%%%%%%%%%%%%%%%
\begin{document}


\section{Grundlagen}
\subsection{Primzahlen}
\subsection{Algebraische Strukturen}
Definiert durch die Zahlentheorie und als zentraler Untersuchungsgegenstand des mathematischen Teilgebietes der universellen Algebra, liefern algebraische Strukturen die Basis zur Realisierung komplexer symmetrischer und asymmetrischer Kryptosysteme, weshalb wir im folgenden Kapitel die Eigenschaften relevanter algebraischer Strukturen näher betrachten wollen. Darüber hinaus möchten wir Ihnen auch einige Werkzeuge zum Rechnen in der jeweiligen algebraischen Struktur an die Hand geben, welche zur späteren Realisierung von Kryptosystemen benötigt werden.

Unter einer sehr allgemeinen Betrachtung ist eine mathematische Struktur eine Liste nichtleerer Mengen, genannt Trägermengen, mit Elementen aus den Trägermengen, genannt Konstanten, und mengentheoretischer Konstruktionen über den Trägermengen. Diese sind konkret Funktionen über den Trägermengen.
Im Weiteren beschränken wir uns auf den Fall einer einzigen Trägermenge, wodurch die Strukturen als homogen bezeichnet werden können.
\paragraph{Definition: Homogene algebraische Struktur}
Eine homogene algebraische Struktur ist ein Tupel $(M,c_1,\dots,c_m,f_1,\dots,f_n)$ mit $m,n \in \mathbb{N}$ und $n \geq 1$. Dabei ist $M$ eine nichtleere Menge, genannt \textbf{Trägermenge}, alle $c_i$ sind Elemente aus $M$, genannt die \textbf{Konstanten}, und alle $f_i$ sind $s_i$-stellige Funktionen $f_i:M \rightarrow M$ im Fall $s = 1$ und $f_i:M^{s_i} \rightarrow M$ im Fall $s_i>1$, genannt die (inneren) \textbf{Operationen}. Die lineare Liste$(0,\dots,0,s_1,\dots,s_n)$ mit $m$ Nullen heißt \textbf{Typ} oder die \textbf{Signatur}.\newline

Laut dieser Definition muss eine homogen algebraische Struktur nicht unbedingt Konstanten enthalten, jedoch mindestens eine Operation. Das Paar $(\mathbb{N},+)$ bildet beispielsweise eine homogene algebraische Struktur des Typs (2). Das 5-Tupel $(\mathbb{N},0,1,+,\cdot)$ bildet ebenfalls eine homogen algebraische Struktur des Typs (0,0,2,2).

Algebraische Strukturen unterscheiden sich grundsätzlich durch ihren Typ. Wirklich charakterisiert werden sie aber erst durch die jeweils geltenden Axiome, d.h. bestimmte Eigenschaften, welche für die Konstanten und Operationen gefordert werden. Durch die Hinzunahme immer weiterer Axiome, entsteht eine Hierarchie immer feinerer Strukturen, an deren Anfang der Monoid steht.

\subsubsection{Monoid}
\paragraph{Definition: Monoid}
Eine algebraische Struktur $(M,e,\cdot)$ des Typs (0,2) heißt ein Monoid, falls für alle $x,y,z\in M$ die folgenden Monoid Aciome gelten: $$x \cdot (y \cdot z) = (x \cdot y) \cdot z$$ $$ e \cdot x = x = x \cdot e$$ Gilt zusätzlich noch für alle $x,y \in M$ die Gleichung $x \cdot y = y \cdot x$, so heiß $(M,e,\cdot)$ ein \textbf{kommutatives Monoid}.

Die erste und die letzte Gleichung bilden das Assoziativ- und Kommutativgesetz ab. Durch die mittlere Gleichung wird ein neutrales Element $e$ bezüglich der Operation gefordert, wobei sowohl die \textbf{Linksneutralität} als auch die \textbf{Rechtsneutralität} spezifiziert wird.

Beispiele für Monoide sind $(\mathbb{N},0,+)$, $(\mathbb{N},1,\cdot)$ und $(\mathbb{Z},0,+)$.
\subsubsection{Gruppe}
\subsubsection{Ring}
\subsubsection{Körper}
\end{document}


